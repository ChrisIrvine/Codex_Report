% Project Proposal Document
\documentclass[review]{cmpreport}
\makeatletter
\input{t1pcr.fd}
\makeatother
\setlength{\footnotesep}{3ex}
\title{\Codex: A Website and Progressive Web App in JavaScript/PHP for table-top role-playing games \\ Living Report}
\author{Christopher Alastair Irvine}
\registration{100036248}
\supervisor{Dr Katharina Huber}
\ccode{CMP-6013Y}
\summary{This document is the Living Report for Codex}
\acknowledgements{
	I would like to thank Dr Katharina Huber for taking on the supervision of this project, 
	and guiding me towards greatness. Additionally I would like to thank Wizards of the Coast 
	for their generosity and kindness in allowing the use of their Intellectual Property for this project.
}
\usepackage{rotating}
\newcommand{\ueacmp}{UEA School of Computing Sciences}
\newcommand{\WotC}{Wizards of the Coast}
\newcommand{\dnd}{D\&D}
\newcommand{\sem}{Software Engineering Model}
\newcommand{\sems}{Software Engineering Models}
\newcommand{\Codex}{\textsc{Codex}}
\newcommand{\AgileSolo}{\emph{Agile Solo}}
% EOF Preamble and Macros
	
% BOF Document
\begin{document}
	% BOF Introduction
	\section{Introduction}
	% Introduce Dungeons and Dragons, when it was created and how the DM has remained at the center of the game.	
	Over the years, \emph{Dungeons and Dragons} (\dnd) have entertained millions of players across the world (\cite{DnDOriginal}). Throughout the various editions of this popular table-top role-playing game, the \emph{Dungeon Master} (DM) has formed the core of every group. The role of a DM is to provide content for the players (party) to explore additionally acting as a; referee, ally, enemy, narrator, historian, peace-keeper and above all friend (\cite{DnDPeople}, \cite{DungeonMaster}). 
	
	The average \dnd \ campaign occurs once a week, for four hours, until the campaign is finished. Typically a DM will spend between 5 and 15 hours preparing content per week, but the work of a DM does not end with their preparation. During the session a DM needs to be alert and attentive to all of their players, reacting accordingly to player input. The vast range of skills required for the role and the amount of time needed to carry out the duties of a DM create a  barrier; preventing many capable players from trying their hand at being a DM, contributing to the shortage of DMs within the community.
			
	\Codex \ aims to make Dungeon Mastering in \dnd \ more accessible and simpler; by reducing the amount of time needed to prepare content and the number of statistics that the DM needs to track and make the general organisation of the group simpler. \Codex \ aims to achieve this through the use of a website that aims to have Progressive Web App (PWA) integration. \dnd \ has been a global community that and a service such as \Codex \ will be of great benefit to them. 
	
	\subsection{A note on Design Documents}
	%Explain and express the virtues of a Design Document.
	The development of \Codex \ is a Software Engineering project which will include a Design Document\footnote{A Design Document is a technical guide for Developers to use before, during and after Developing a piece of Software.}(\cite{DesignDocExample}), allowing the solutions to be planned out and tested before development begins.
	
	\subsection{Challenges and Strategies}
	We anticipate that \Codex \ will face a several challenges over the course of its development, strategies have been outlined below to deal with each of the foreseeable challenges.	

	\subsubsection{Choosing a \sem \ (SEM)} 
	%Application of Software Engineering Models & Software Design
	The most extensive problem that influences \Codex \ is the application of SEM to the development of \Codex, which will affect the structure of the Software. As the majority of deliverables are non-programming, the chosen SEM will need to provide advantages that can be applied to all tasks. Nyst{\"o}m details the development and testing of a SEM, called \AgileSolo \ (\cite{AgileSolo}). \AgileSolo \ provides many time keeping and review benefits that can be applied to all tasks. If \AgileSolo \ is unsuitable for \Codex, a substitute SEM can be used called \emph{ExtremeProgramming} (XP). XP can be adapted for a Single Developer Project \ (\cite{SoloXP}).
	
	\subsubsection{GUI Design, Development and Testing}
	% Handling GUI Design.
	As the focus of \Codex \ is to reduce the burden of being a DM, it is vital that the GUI is well designed and easy to use. Galitz outlines the principle of good GUI design in this book (\cite{GUIDesign}), which will serve as a guide for developing and testing the GUI for \Codex.
	
	\subsubsection{Client- and Server-Side Technologies}
	% Division of Server-side and Client-side Technologies
	Maintaining a good balance of Server-side and Client-side technologies ensures the fastest load time possible. Server-side functions will complete on the server prior to the page loading, whereas Client-side functions will complete on the end user's machine after the page is loaded.
	
	\subsubsection{Offline Capabilities and Caching}
	% Offline Capabilities & Temporary Storage
	Client-side technologies have the unique advantage of being able to run Offline, given the necessary data. Once the original website based application has been developed, \Codex \ will be converted to a PWA by JavaScript libraries - such as NodeJS. The PWA version of \Codex \ will come with Offline capabilities that are defined within the Design Document.
	
	\subsubsection{Database Design, Development and Management}
	% Development, design and management of a Server-side Database.
	The database is hosted on a server and will be built in MySQL. Database communication is handled through a PHP or JavaScript back-end. The design of the database will be mapped out in the Design Document, so that a test database can be built to ensure integrity. 
	
	\subsubsection{Algorithm Design}
	% Algorithm Design
	\Codex \ contains a multi-dimensional algorithm that will provide a DM with an accurate difficulty score for a fight. The original \emph{Challenge Rating} system, detailed in the Monster Manual \citep{MonsterManual} and Volo's Guide to Monsters  \citep{Volos}, will be used so that earlier features can function until the difficulty algorithm can be developed. 
	
	\subsubsection{Gathering Data from the \dnd \ Player Base}
	% Gathering data from D&D player base
	DMs are scattered across the globe; but via circulating a survey through the use of Social Media, \Codex \ could ascertain the habits of the DM populace for the first time. Improving the features of \Codex \ during the design phase of the project.
	
	\subsection{Resources}
	% "Have an appreciation of the resources that are likely to be required to ensure completion"
	We anticipate that two resources are critical to the success of \Codex. Firstly, the attainment of suitable server space, possibly purchased through companies such as DigitalOcean. Additionally the Ethics Approval, provided by the Ethics Board, for the Survey before it can be distributed. 
	
	\clearpage
	
	\section{Literature Review}
	In this document we will explore academic journals, papers and books that are relevant to the development of \Codex, in addition we will gain a deeper appreciation for the source material that govern the 5th Edition of \dnd \ (see Section \ref{DnDRules&Lit}). Topics that are covered in this literature review include; Software Engineering Models (see Section \ref{SEMLit}), Web-App Architecture (see Section \ref{Web-Arch}), Offline Capabilities (see Section \ref{Offline}), Database Design (see Section \ref{Database}), Graphical User Interface (GUI) Design and Testing (see Section \ref{GUIs}). Other topics will be explored in addition to those listed.
	
	\subsection{\Codex \ and Dungeons and Dragons} \label{DnDRules&Lit}
	\Codex \ is a tool for the DM of a group to use in order that they can enjoy running games of \dnd \ to a higher degree. To do this, a limit has been set on\Codex \ scope. \Codex \ will not replace any dice rolling or the simple arithmetic that fuels the game. Instead \Codex \ will be an interface to track game statistics and quickly review rules and ability effects,  allowing the DM to instil more narrative and creative content into the game.
	
	\subsubsection{Understanding Dungeons and Dragons}
	\dnd \ is a game that is random by design, but there are two distinct areas to \dnd \, \emph{Exploration} and \emph{Combat}. \Codex \ will largely deal with the Combat rules, however there is scope to support the DMs in the Exploration area as well. There is one overarching concept about \dnd \ that needs to be understood when dealing with the rules of game. The rules are nothing more than a suggestion to the DM about how to run his or her game. Every group will have their own way of dealing with different situations, which could be an entirely new rule or an adjustment to one of the suggested rules. \dnd \ is governed by the roll of dice. The most common die that is used is the 20-sided die, which dictates if an attack hits, or if you were able to persuade a character in game. It is used in both Combat and Exploration. The other dice (100, 12, 10, 8, 6 and 4 sided) are largely used for damage or the result of a spell. These dice are denoted by a lower-case `d' and the number of side that dice posses, for example the 20-sided dice are denoted as \textbf{d20}.
	
	For a worked example of \dnd \ please see Appendix \ref{DnDExample}.
	
	\subsubsection{Exploration}
	We shall begin with a brief look into Exploration, this is where the majority of DM preparation time is spent. DMs need to have landscapes, settlements, characters, dialogue, plot points and consequences for Players to interact with whilst they explore the world. The Dungeon Master's Guide \citep{DMGuide} contains a lot of information about the officially published settings, as well as tips and tricks for creating your own adventures. One of the most useful sources in the Dungeon Master's Guide for creating adventures for the Players to explore, is the multitude of random chance tables that allow you to create all the necessary information that forms the core of any adventure you might need for your game. These random chance tables are not limited to adventure story lines, other tables cover the characters the Players may interact with or the treasures players might receive at the end of an adventure. \Codex \ aims to include a planning feature that will integrate these tables to allow the busy DM to instantly create the framework for adventures and characters. 
	
	\subsubsection{Combat}
	There are a lot more combat rules than exploration rules spread throughout the three \dnd \ core books (\cite{DMGuide}, \cite{MonsterManual} and \cite{PlayerHandbook}) . These rules are also more heavily suggested towards the DM as quite often the lives of the Player's Characters depends on these rules. Some DMs do choose to alter these rules to suit their particular game \citep{Personal}, however \Codex \ will follow these rules as written. The Combat rules are the same for both the Player Characters and the enemies they face, and can be broken down into a set of simple calculations. For example, when seeing if an attack with a hammer will hit; the character's \textbf{proficiency} \ plus \textbf{strength modifier} \ will be added to the result of a \textbf{d20 roll}. This can be summarised to be:

	The attack roll is 23, this is then compared against the targets \textbf{Armour Class (AC)}. Plate Armour has an AC of 18, so the attack roll of 23 is greater than the AC of the target and will hit \citep{PlayerHandbook}. This rule will not be included in \Codex's scope as it is taking away enjoyment from the players and is not helping the DM do their job. Instead it is rules like the \emph{poisoned} rule that will be taken into account, where a suitable icon will be placed over the character who is \emph{poisoned}, to remind the DM that  character has the condition.
	
	\subsection{Software Engineering Models} \label{SEMLit}
	A \sem at its core, is a model for time management in relation to a project, particularly the Agile \sems \ - such as Scrum  \citep{Scrum} . The majority of \sems \ are focused on controlling development time and code reviews, however some can extend to managing research and design work as well. For \textsc{Codex} we will look at two potential \sems \ to support the research, design and development phases. 
	 
	\subsubsection{Agile for One}
	Agile for One is an adaptation of the Agile Manifesto to be more suitable for a single developer project, with the outcome of defining a new \sem , called Agile Solo \citep{AgileSolo}. Agile Solo is similar to traditional Scrum in that there are weekly iterations inside a longer over-arching period of time, but rather than a two week sprint encapsulating a daily scrum meeting, there is a monthly deliverable with a weekly iteration. At the end of each week, the Developer meets with the Customer or a Supervisor to discuss what occurred during the week and what the next focus should be. A longer review takes place at the end of each month. Agile Solo also recommends a daily meeting with a fellow developer to ensure that code quality is maintained. 
	
	\subsubsection{Extreme Programming For One}
	Extreme Programming (XP) is a \sem \ which emphasises the importance of constant feedback and maintaining simplicity in a system, allowing a project to thrive in an ever-changing environment. XP For One preserves the traditional stages of XP, but the \textbf{Pair Programming} recommendation is impossible to replicate in a solo developer project \citep{SoloXP}. So instead XP For One suggests; that the developer has a friend or colleague constantly check that the developer has run the tests on the current section of work, that a log book is kept detailing the tests and development results and that the developer meet with a supervisor or client regularly to maintain the constant feedback. 
	
	\subsubsection{Development of \Codex}
	The development of \Codex, which includes the non-programming activities (such as reports and presentations) will be achieved by utilising Agile for One Software Engineering Model for all aspects of the report. Trello, a website dedicated to project management, will aid in the management of tasks \citep{Trello}. Should Agile for One be unsuitable for the development of \Codex \ then Extreme Programming For One will be used. 
	
	\subsection{Web-App Architecture} \label{Web-Arch}
	Web-App Architecture can be simplified down to three layers [INSERT IMAGE OF ARCHITECTURE FROM \citep{SecurityWebApps} HERE, LABELLING IT FIGURE 2]. As we can see from Figure 2 the first layer (Presentation Tier) is held locally on the Client's machine. This is what is known as `Client-Side'. Here HTML, CSS and JavaScript is used to translate data into a format the Client will understand. The second layer (Logic Tier), here languages such as PHP or Node.JS perform functions on raw data to generate web content. Traditionally the Logic Tier is hosted on a Server that Clients connect to, this is the `Server-Side' The third layer (Data Tier) communicates with the Logic Tier through SQL Queries \citep{SecurityWebApps}.
	
	\subsubsection{Server- vs Client-Side Technologies}
	From Figure 2, we can create a rule for the division of \Codex' functions between Server- and Client-Side technologies. All data and logical functions should be hosted on the Server-Side and the display functions should be hosted on Client-Side. However, this does create a problem. This rule means that \Codex \ will have no Offline Functionality. 
	
	\subsection{Offline Capabilities} \label{Offline}
	

	\subsection{Databases} \label{Database}
	\subsubsection{Database Design}

	\subsection{Graphical User Interface} \label{GUIs}
	\subsubsection{GUI Design}
	\subsubsection{GUI Testing}
	\subsubsection{Mobile GUI Design and Testing}
	
	\begin{cmpfigure}{\Codex \ Gantt Chart, outlining the major tasks and deliverables\label{pplan}}
		\begin{sideways}
			\newganttchartelement{voidbar}{
				voidbar/.style={draw=black, top color=black!25, bottom color=black!23
			}}
			\begin{ganttchart}[y unit chart = 0.86cm, y unit title = 0.86cm, x unit=0.45cm, vgrid, title label font=\scriptsize,
				canvas/.style={draw=black, dotted}]{1}{34}
				\gantttitle{Project schedule shown for e-vision week numbers
					and semester week numbers}{34} \\
				
				\gantttitlelist{8,...,41}{1}\\
				\gantttitlelist{1,...,12}{1} \gantttitle{CB}{4}
				\gantttitlelist{1,...,9}{1} \gantttitle{EB}{4}
				\gantttitlelist{10,...,14}{1}\\
				
				%The elements, bars and milestones, are identified as elem0, elem1, etc.
				\ganttbar{Project Proposal}{1}{2} \\        		%elem0  
				\ganttbar{Literature Review}{2}{5} \\      			%elem1 
				\ganttmilestone{Literature Review Finished}{5} \\	%elem2
				\ganttbar{Design Doc. Iteration 1}{4}{8} \\ 		%elem3
				\ganttbar{Development Iteration 1}{9}{11} \\		%elem4
				\ganttbar{Progress Report}{11}{12} 					%elem5
				\ganttmilestone{}{12} \\							%elem6
				\ganttbar{Development Iteration 2}{11}{12}  		%elem7
				%week 1 of semester 2 is the 17th week in schedule 
				\ganttvoidbar{}{13}{16} \\                     		%elem8
				\ganttbar{Final Report Writing 1}{17}{19} 	        %elem9
				\ganttmilestone{}{19} \\							%elem10
				\ganttbar{Design Doc. Iteration 2}{19}{20} \\      	%elem11
				\ganttbar{Development Iteration 3}{21}{26} \\		%elem12
				\ganttbar{Testing}{24}{26} \\						%elem13
				\ganttmilestone{Code Delivery}{26} \\       		%elem14
				\ganttbar{Final Report Writing 2}{27}{30} \\  		%elem15
				\ganttbar{Inspection preparation}{31}{34}   		%elem16
				
				\ganttlink{elem0}{elem1} \ganttlink{elem1}{elem2} \ganttlink{elem2}{elem3}
				\ganttlink{elem3}{elem4} \ganttlink{elem4}{elem5} \ganttlink{elem5}{elem6}
				\ganttlink{elem5}{elem7} \ganttlink{elem8}{elem9} \ganttlink{elem10}{elem11}
				\ganttlink{elem11}{elem12} \ganttlink{elem12}{elem13} 
				\ganttlink{elem13}{elem14} \ganttlink{elem14}{elem15}
				\ganttlink{elem15}{elem16}
			\end{ganttchart}
		\end{sideways}
	\end{cmpfigure}
	\clearpage
	\appendix
	\section{Dungeons and Dragons 5th Edition Worked Example} \label{DnDExample}
	This is a simplified example of a Dungeon Master preparing and running a session of Dungeons and Dragons. There are a lot of intricate details that are left out on purpose. By the end of this Appendix, the reader should understand the basics of Dungeons and Dragons. 
	
	\subsection{Preparing a Session} \label{DnDPrepExample}
	Adam is a Dungeon Master (DM), he runs games of 5th Edition \dnd \ for his friends. Together they have been playing for over a year in a campaign based in an altered version of the Forgotten Realm Setting. Last week Adam ended the session on a cliff-hanger moment where one of his players, Stephanie, was captured by the recurring Villain of the Campaign - Warlord Trinton. For the next session Adam has to think about what would Warlord Trinton do with Stephanie's character, provide the opportunity for the rest of the Party to rescue Stephanie as well as the chance for Stephanie to escape by herself.
	
	Those are the three key points of the next session. Outside of that Adam will have to prepare some Dialogue notes for Warlord Trinton, as well as information for the party to find about Stephanie's location. Depending on how much Adam has prepared for this situation in the past, he could be spending over 10 hours preparing for the next session of his Campaign. Adam manages to find enough time to prepare the necessary information he needs to breathe life into the world and carry on the Campaign over the week, often preparing after work in the evenings. This week Adam spent 15 hours preparing for his next session of \dnd. 
	
	\subsection{Running a Session} \label{DnDSeshExample}
	With his players; Stephanie, Megan, John and Philip, gathered around the kitchen Table Adam begins this weeks game of \dnd \ beginning with a recap of the events of last session. Megan, John and Philip recover from their wounds they got fighting Warlord Trinton and set off after Stephanie. They travel to a nearby village and ask the locals if they have seen Warlord Trinton pass through, with a Cart in tow behind him and his minions. Adam had anticipated this eventuality, so he had Trinton parade Stephanie's Character through the Village so that all could of seen him pass. But he was not going to give the information up for free. Adam, role-playing a Farmer in the village converses with the party:
	
	\textsc{Megan}: Excuse me Sir, have you seen Warlord Trinton and his minions drag a cart through here recently?
	
	\textsc{Adam}: We should not be discussing such things, the Warlord's spies might hear! He does not care about us Peasants.
	
	\textsc{Megan}: The Warlord has taken our friend, we are out to defeat him and release this region from his grasp. All we need is a yes or no and then you we wont bother you again.
	
	Adam recognises Megan's last piece of dialogue as an attempt to persuade the Farmer to give up the information. So he asks her to make a \textbf{Persuasion Roll}. Megan rolled a 20 sided dice (she got a 9) and added her +7 proficiency modifier totalling 16. It was not a difficult task to persuade the Farmer, Adam had wrote this to be 10 Difficulty in his notes, meaning that the \textbf{Persuasion Check} would need to be 10 or higher to succeed.
	
	\textsc{Adam}: He was heading up the road, toward his Black Tower. 
	
	\textsc{Megan}: Thank you, we will ensure that you and your family are not caught up in all of this.
	
	Adam decides to check in on Stephanie so that she is not bored at the table waiting to be rescued. There is a brief exchange of words between Stephanie and Warlord Trinton, before Trinton goes to attend to his other matters. Stephanie's Character is being taken down to the dungeons below The Black Tower. Role-Playing her Goliath Barbarian character, she decides not to wait to be rescued. Instead she waits until she is deep within dungeons and is sure that there are only the two minions escorting her. She declares to Adam that she is going to attempt to free herself by throwing one of the minions down into a hole that a spiral staircase is circling (taking them into the dungeons) and fight the other. She passes her \textbf{Strength Check} to throw a minion into the hole, now she has to fight the minion. With initiative order worked out, the minion attacks first. Mathematically this is how this fight would look, for the purposes of this example, both combatants will use the same formulae:
	\[d20 + strength \ modifier + proficeincy = attack \ roll (vs \ targets \ Armour \ Class)\]
	\[d6 + strength \ modifier = damage\]
	\[Minion \ Attack \ Roll: 8 + 4 + 2 = 14 \ vs \ 15 ... \ attack \ misses\]
	\[Stephanie \ Attack \ Roll: 15 + 4 + 3 = 21 \ vs \ 14 ... \ attack \ hits\]
	\[Stephanie \ Damage \ Roll: 4 + 4 = 8 \ damage \ to \ the \ Minion\]
	\[Minion \ Attack \ Roll: 18 + 4 + 2 = 24 \ vs \ 15 ... \ attack \ hits\]
	\[Minion \ Attack \ Roll: 2 + 4 = 6 \ damage \ to \ Stephanie's \ Character\]
	This pattern carries on until Stephanie eventually defeats the Minion. Now Stephanie is faced with a new problem, how to escape from Warlord Trinton's Black Tower.
	
	\subsection{Conclusion to Worked Example}\label{DnDWEConc}
	That content in Appendix \ref{DnDSeshExample} would roughly be 20-30 minutes of session time, the session would of carried on for another three hours at the very least. Whilst this is a lot of work for Adam, he genuinely enjoys the feeling of providing this living, breathing world for his friends to explore and get themselves into many misadventures.
	
	\clearpage
	
\bibliography{projectbib}
\end{document}
% EOF Document